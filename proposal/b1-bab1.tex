%==================================================================
% Ini adalah bab 1
% Silahkan edit sesuai kebutuhan, baik menambah atau mengurangi \section, \subsection
%==================================================================

\chapter[PENDAHULUAN]{\\ PENDAHULUAN}

\section{Latar Belakang}
Industri makanan dan minuman (\textit{Food and Beverage}/F\&B) merupakan salah satu sektor yang sangat dinamis dan memiliki tingkat kompetisi yang tinggi. Dalam proses pengambilan keputusan, konsumen modern tidak lagi hanya mengandalkan pengalaman langsung, tetapi juga mempertimbangkan berbagai ulasan daring yang tersedia pada platform seperti \textit{Google Review}, \textit{Instagram}, dan \textit{GoFood Review}. Ulasan tersebut menjadi indikator penting bagi perusahaan untuk memahami persepsi pelanggan terhadap kualitas produk dan layanan yang diberikan. Hal ini sejalan dengan temuan berbagai penelitian yang menunjukkan bahwa opini pengguna pada platform daring sangat berpengaruh dalam membentuk citra dan reputasi suatu layanan atau institusi \cite{sejati_aspect-based_2024}. Dalam konteks yang lebih luas, ulasan konsumen telah dimanfaatkan secara efektif pada sektor pariwisata, kesehatan, dan layanan publik untuk menganalisis pengalaman pengguna serta mengevaluasi kualitas layanan berbasis data. Oleh karena itu, ulasan daring memiliki peran strategis, khususnya bagi industri F\&B, dalam menyusun rencana perbaikan layanan secara berkelanjutan.\\
\indent Meskipun demikian, volume ulasan yang besar serta karakter teks yang tidak terstruktur menimbulkan tantangan yang signifikan. Mayoritas ulasan konsumen ditulis dalam bentuk naratif bebas dengan variasi bahasa yang tinggi, mulai dari bahasa informal, singkatan, hingga penggunaan bahasa campuran (\textit{code-mixing}). Kondisi ini menyebabkan analisis manual menjadi tidak efisien dan rentan terhadap bias. Penelitian oleh \cite{dhendra_benchmarking_2025} menjelaskan bahwa data ulasan dalam jumlah besar memiliki tingkat keragaman yang tinggi sehingga pendekatan manual tidak mampu memberikan gambaran yang komprehensif secara cepat dan akurat. Selain itu, karakteristik Bahasa Indonesia pada media sosial cenderung tidak mengikuti kaidah bahasa formal, sehingga memerlukan pendekatan \textit{Natural Language Processing} (NLP) yang lebih adaptif terhadap ragam bahasa tersebut.\\
\indent Analisis sentimen merupakan salah satu teknik NLP yang banyak digunakan untuk mengekstraksi emosi atau opini yang terkandung dalam teks. Teknik ini telah terbukti efektif dalam berbagai penelitian, seperti pada ulasan layanan kesehatan \cite{maretta_aspect-based_2025}, transportasi publik, hingga pendidikan. Namun, analisis sentimen konvensional umumnya hanya mampu mengklasifikasikan sentimen secara umum pada tingkat kalimat atau dokumen. Pendekatan ini menjadi kurang memadai ketika sebuah ulasan mengandung beberapa opini berbeda terhadap berbagai aspek dalam satu produk atau layanan. Sebagai contoh, konsumen dapat menyatakan bahwa rasa makanan enak (sentimen positif pada aspek kualitas makanan), tetapi pelayanannya lambat (sentimen negatif pada aspek pelayanan), serta harga yang dianggap terlalu mahal (sentimen negatif pada aspek harga). Kondisi serupa juga ditemukan pada penelitian lain, seperti ulasan hotel atau aplikasi kesehatan, di mana satu ulasan dapat mencakup berbagai aspek yang memerlukan analisis secara terpisah \cite{singgalen_performance_2025}.\\
\indent Untuk mengatasi keterbatasan tersebut, dikembangkan pendekatan \textit{Aspect-Based Sentiment Analysis} (ABSA). Berbeda dengan analisis sentimen konvensional, ABSA memecah teks ulasan ke dalam aspek-aspek tertentu, kemudian mengidentifikasi sentimen pada masing-masing aspek tersebut. Berbagai penelitian menunjukkan bahwa ABSA mampu menghasilkan analisis yang lebih mendalam dan akurat. Penelitian oleh \cite{perwira_domain-specific_2025}, misalnya, menunjukkan bahwa penerapan ABSA pada ulasan pariwisata dapat mengidentifikasi aspek-aspek penting secara otomatis dengan tingkat akurasi sebesar 84\%. Penelitian lain oleh \cite{maretta_aspect-based_2025} dalam domain kesehatan juga membuktikan bahwa ABSA berbasis \textit{transformer} mampu mencapai akurasi hingga 96\% dengan nilai \textit{macro F1-score} sebesar 0,90. Pada sektor pendidikan, \cite{jazuli_optimizing_2024} melaporkan bahwa model IndoBERT dengan pendekatan \textit{fine-tuning} mampu menghasilkan akurasi hingga 97,9\% dalam tugas ABSA. Temuan-temuan tersebut menegaskan pentingnya ABSA sebagai pendekatan yang lebih tepat dalam menganalisis ulasan dengan kompleksitas multiaspek.\\
\indent Namun demikian, penerapan ABSA pada Bahasa Indonesia memiliki tantangan tersendiri akibat kompleksitas linguistiknya. Bahasa Indonesia memiliki fleksibilitas sintaksis, tingginya penggunaan bahasa informal, serta variasi dialek yang menyebabkan model NLP berbahasa Inggris maupun model \textit{multilingual} kurang optimal. Model \textit{multilingual} seperti \textit{mBERT} dan \textit{XLM-R}, misalnya, dilaporkan memiliki performa yang lebih rendah dibandingkan model yang dilatih secara spesifik untuk Bahasa Indonesia \cite{dhendra_benchmarking_2025}. Oleh karena itu, diperlukan model yang mampu memahami karakteristik linguistik Bahasa Indonesia secara lebih mendalam.\\
\indent Menjawab kebutuhan tersebut, \cite{wilie_indonlu_2020} mengembangkan IndoBERT, yaitu model berbasis arsitektur \textit{Bidirectional Encoder Representations from Transformers} (BERT) yang dilatih menggunakan korpus besar yang terdiri dari lebih dari 4 miliar token Bahasa Indonesia. IndoBERT dirilis melalui proyek \textit{IndoNLU} dan menjadi model dasar yang banyak digunakan dalam berbagai penelitian NLP berbahasa Indonesia. IndoBERT telah terbukti unggul dalam berbagai tugas, seperti klasifikasi sentimen, analisis emosi, ekstraksi aspek, hingga \textit{aspect-based opinion mining}. Penelitian oleh \cite{dhendra_benchmarking_2025} menunjukkan bahwa IndoBERT mencapai akurasi tertinggi sebesar 88,1\% dalam klasifikasi sentimen ulasan layanan \textit{e-government} dibandingkan dengan \textit{mBERT} dan \textit{XLM-R}. Sementara itu, \cite{jazuli_optimizing_2024} melaporkan peningkatan signifikan dalam akurasi ABSA pada ulasan mahasiswa dengan nilai \textit{F1-score} mencapai 0,974. Selain performanya yang unggul, IndoBERT juga lebih adaptif terhadap variasi Bahasa Indonesia, baik formal maupun informal.\\
\indent Berdasarkan latar belakang tersebut, penelitian ini mengusulkan pengembangan sistem ABSA berbasis IndoBERT untuk menganalisis ulasan konsumen pada industri F\&B dengan fokus pada tiga aspek utama, yaitu kualitas makanan, pelayanan, dan harga. Penelitian ini tidak hanya berfokus pada pemodelan dan proses \textit{fine-tuning} IndoBERT, tetapi juga menekankan tahap implementasi melalui penyediaan layanan \textit{Application Programming Interface} (API) agar dapat diintegrasikan dengan sistem eksternal seperti dashboard analitik atau aplikasi berbasis web. Dengan demikian, perusahaan F\&B dapat memperoleh \textit{insight} sentimen yang lebih presisi dan terperinci berdasarkan aspek, sehingga pengambilan keputusan dapat dilakukan secara lebih tepat dan berbasis data. Selain itu, penelitian ini diharapkan dapat berkontribusi dalam pengembangan NLP Bahasa Indonesia dengan menyediakan model ABSA yang adaptif terhadap karakteristik ulasan konsumen pada sektor F\&B.

\section{Rumusan Masalah}
\indent Berdasarkan latar belakang di atas, rumusan masalah dalam penelitian ini adalah sebagai berikut:
\begin{enumerate}
    \item Bagaimana menerapkan teknik \textit{fine-tuning} pada model IndoBERT untuk meningkatkan akurasi analisis sentimen berbasis aspek pada ulasan konsumen sektor F\&B?
    \item Bagaimana merancang proses \textit{deployment} model IndoBERT hasil \textit{fine-tuning} ke dalam bentuk layanan \textit{API} yang efisien dan dapat diintegrasikan dengan aplikasi \textit{website} analisis sentimen?
    \item Bagaimana memastikan layanan \textit{API} yang dibangun mampu memberikan analisis sentimen untuk tiga aspek utama, yaitu kualitas makanan, pelayanan, dan harga?
\end{enumerate}

\section{Tujuan Proyek}
\indent Tujuan dari penelitian ini adalah sebagai berikut:
\begin{enumerate}
    \item Menerapkan teknik \textit{fine-tuning} pada model IndoBERT untuk tugas \textit{Aspect-Based Sentiment Analysis} (ABSA) pada ulasan konsumen sektor F\&B.
    \item Melakukan \textit{deployment} model hasil \textit{fine-tuning} ke dalam bentuk layanan \textit{API} yang dapat digunakan oleh aplikasi \textit{website} analisis sentimen.
    \item Mengembangkan sistem \textit{API} yang mampu menerima input berupa teks ulasan dan menghasilkan klasifikasi sentimen berdasarkan tiga aspek utama dengan kategori positif, negatif, dan netral.
\end{enumerate}

\section{Manfaat Proyek}
\begin{enumerate}
    \item \textbf{Bagi Industri F\&B:}\\
    Memberikan solusi praktis bagi perusahaan dalam menganalisis persepsi pelanggan secara otomatis, sehingga dapat dimanfaatkan untuk meningkatkan kualitas produk dan layanan.
    \item \textbf{Bagi Pengembangan Teknologi:}\\
    Memberikan kontribusi dalam penerapan \textit{Large Language Model} berbasis IndoBERT untuk Bahasa Indonesia serta implementasi \textit{deployment pipeline} menggunakan layanan \textit{API}.
    \item \textbf{Bagi Akademisi dan Peneliti:}\\
    Menjadi referensi bagi pengembangan penelitian lanjutan di bidang ABSA, khususnya terkait penerapan model \textit{transformer} dan integrasinya dalam sistem berbasis layanan daring.
\end{enumerate}

\section{Batasan Proyek}
\begin{enumerate}
    \item \textbf{Aspek Analisis:}\\
    Penelitian ini difokuskan pada tiga aspek utama yang relevan dalam sektor F\&B, yaitu kualitas makanan, pelayanan, dan harga. Aspek lain di luar ruang lingkup penelitian, seperti suasana, lokasi, dan kebersihan tempat, dikelompokkan ke dalam aspek pelayanan.
    \item \textbf{Sumber Data Ulasan:}\\
    Data penelitian dibatasi pada ulasan \textit{Google Maps Reviews} hasil proses \textit{scraping} dari lima gerai F\&B yang berlokasi di wilayah Jawa Timur, yaitu Bakso Sayur UB, Kopi Studio 24, Geprek Kak Rose, Mie Gacoan, dan Sego Tempong Mbok Nah.
    \item \textbf{Bahasa Teks:}\\
    Seluruh analisis dilakukan pada ulasan berbahasa Indonesia. Ulasan dengan dominasi bahasa non-Indonesia dikeluarkan dari dataset untuk menjaga konsistensi linguistik.
    \item \textbf{Model dan Framework:}\\
    Penelitian ini menggunakan model IndoBERT-base-1 dengan pendekatan \textit{fine-tuning} untuk tugas ABSA tanpa melakukan perbandingan dengan model lain seperti \textit{mBERT} atau \textit{XLM-R}.
    \item \textbf{Kategori Sentimen:}\\
    Kategori sentimen dibatasi pada tiga kelas, yaitu positif, negatif, dan netral.
    \item \textbf{Implementasi Sistem:}\\
    Implementasi penelitian ini terbatas pada pembangunan dan \textit{deployment} model dalam bentuk layanan \textit{RESTful API} tanpa pengembangan antarmuka pengguna.
\end{enumerate}